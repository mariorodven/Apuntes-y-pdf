\documentclass{article}
\usepackage{amssymb}
\usepackage{tikz}
\usepackage{pgfplots}
\usepackage{amsmath}
\pgfplotsset{compat=1.16}
\usepackage[a4paper, total={6in, 8in}]{geometry}

\title{Lección 2: Ecuaciones lineales de segundo orden}
\author{Mario Rodríguez}
\date{Feb 2023}

\begin{document}

\maketitle

\section{Definicion}
Llamamos ecuación diferencial de segundo orden a una igualdad de la forma:
\[y''+p(x)\cdot y'+q(x)\cdot y =r(x)\]
Además, $p(x), q(x) , r(x) \in I$ y son contínuas en $I$.\\
Diremos que la ecuación es homogenea si la función $r(x) = 0$, quedando así la ecuación:
\[y''+p(x)\cdot y'+q(x)\cdot y =0\]
Adicionalmente, diremos que la ecuación es de coeficientes constantes si $p(x)\in \mathbb{R}$ y $q(x)\in \mathbb{R}$ queda tal que asi:
\[y''+p\cdot y'+q\cdot y= r(x)\]
\[y''+p\cdot y'+q\cdot y= 0\]

\subsection[P.V.I]{Definicion de problemas de valores iniciales}

\[y'' +p(x)\cdot y' + q(x)\cdot y = r(x) \; x\in I\] 

\end{document}
