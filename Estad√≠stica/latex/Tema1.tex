\documentclass{article}
\usepackage{amssymb}
\usepackage{tikz}
\usepackage{pgfplots}
\usepackage{amsmath}
\pgfplotsset{compat=1.16}
\usepackage[a4paper, total={6in, 8in}]{geometry}

\title{Tema 1: Introducción a la teoría de la probabilidad}
\author{Mario Rodríguez}
\date{Feb 2023}

\begin{document}

\maketitle
\section{Conceptos básicos}
\subsection{Experimento aleatorio, muestral y sucesos}
Experimento aleatorio es aquel del que a priori no conocemos el resultado, sin embargo siconocemos todos los posibles resultados.\\

Experimento aleatorio $\equiv$ {resultados posibles} = $\Omega$: Espacio muestral de $\Omega$
\\
\\En un dado de 4 caras por ejemplo: $\Omega = {1,2,3,4} \rightarrow E={{\emptyset},{1},{2},{3},{4}\ldots{\Omega}}$
\\Otro ejemplo sería por ejemplo el tiempo de vida de una bombilla: $\Omega = [0,+\infty]=\mathbb{R}\textsuperscript{+}$
\\
\\Decimos que $\Omega$ es discreto si es finito o numerable infinito. A cada subconjunto de $\Omega$ se le llamará suceso simple si es un único elemento o compuesto si tiene mas de un elemento. Dentro de los sucesos tenemos a $\Omega$, que representa el suceso seguro y a $\emptyset$ que representa el suceso imposible.
\\


\end{document}